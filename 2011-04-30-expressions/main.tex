\documentclass{beamer}
\usepackage{minted}
\usemintedstyle{vs}

\title{
	Functional composition in C\#\\
	Expression trees and dynamic (re)compilation
}
\author{Matt Enright \and {John Campbell}}
\date[]{
	Pittsburgh CodeCamp 2011.1\\
	30. April 2011
}

\AtBeginSection[] {
	\frame{\sectionpage}
	\frame{\tableofcontents[currentsection]}
}

\begin{document}
\frame[plain]{\titlepage\includegraphics[scale=0.5]{snap}}

\begin{frame}[plain]
	\frametitle{Abstract}
	.NET 3.5 added powerful new capabilities for both code generation and
	function manipulation. Expression syntax has now entered the mainstream
	via the use of lambda expressions in APIs like LINQ, Moq, and ASP.NET MVC
	as a way to reproduce reflection capabilities in a compile-safe fashion --
	but reflection is just scratching at the surface of the power of
	expression trees. This talk will look at what these APIs are doing under
	the hood, and then dive into dynamic expression generation in complex
	software like LINQ providers and the DLR, plus exploring the new
	expression APIs for control structures and statements in .NET4 and showing
	some novel uses of expression trees to do the otherwise-impossible in
	C\#.
\end{frame}
\frame[plain]{\tableofcontents}

\section{Introduction}
\subsection{About us}
\subsection{Delegates}
	\begin{frame}
		"function pointer"
	\end{frame}
	\begin{frame}
		Delegate instance/object/type
		Multi-cast?
		Assignation versus invocation
	\end{frame}
	\begin{frame}
		Anonymous delegate
	\end{frame}
	\begin{frame}
		CLOSURE
	\end{frame}
	\begin{frame}
		Lambda expression syntax
		type inference
	\end{frame}

\subsection{Expressions}
	\begin{frame}
		Definition
		Value, type
	\end{frame}
	\begin{frame}
		Immutability
	\end{frame}
	\begin{frame}
		Difference to delegate instance
		Same syntax (lambda)
	\end{frame}
	\begin{frame}
		System.Linq.Expressions.Expression
		\begin{itemize}
			\item Name
			\item Reduction (CanReduce/Reduce ())
		\end{itemize}
	\end{frame}

\section{Expression syntax}
\subsection{API design}
	\begin{frame}
	\end{frame}
	\begin{frame}[fragile]
		\frametitle{Code as data}
		\begin{minted}[fontsize=\scriptsize]{csharp}
		static string MakeLabel (LambdaExpression expression)
		{
		    if (expression.Body.NodeType == ExpressionType.MemberAccess)
		        return ((MemberExpression)part).Member.Name;

		    if (expression.Body.NodeType == ExpressionType.ArrayIndex) {
		        var lambda = Expression.Lambda<Func<object,object>> (
		            Expression.Convert (expression, typeof(object)),
		            Expression.Parameter (typeof(object), null));

		        return Convert.ToString (lambda.Compile ().Invoke (null));
		    }

		    return String.Empty;
		}
		\end{minted}
	\end{frame}

\subsection{Reflection}
	\begin{frame}[fragile]
		\begin{minted}{csharp}
		var myString = "PGHdotNET";

		object lengthProp = typeof(string).GetProperty ("Length");
		var name = lengthProp.Name;
		int len = (int)info.GetValue (myString);
		\end{minted}
	\end{frame}

	\begin{frame}
		\Huge That sucks.
	\end{frame}

	\begin{frame}[fragile]
		\frametitle{}
		\begin{itemize}
			\item Not compile-safe.
				\pause
				\begin{minted}[fontsize=\scriptsize]{csharp}
				PropertyInfo lengthProp = typeof(string).GetProperty ("Lengh");
				int len = (int)info.GetValue ("hello");
				\end{minted}
				\pause
			\item Not readable.
				\pause
				\begin{minted}[fontsize=\scriptsize]{csharp}
				PropertyInfo lengthProp = typeof(string).GetProperty ()
				\end{minted}
				\pause
			\item Not writeable
				\pause
				\begin{minted}[fontsize=\scriptsize]{csharp}
				MethodInfo equals = typeof(string).GetMethod ("Equals");
				\end{minted}
				\pause
			\item Slow. \pause
		\end{itemize}
	\end{frame}

	\begin{frame}
	\end{frame}

	\begin{frame}[fragile]
		\frametitle{Example: Generic operators}
		\begin{minted}{csharp}
		static T Add<T> (T a, T b)
		{
	    	var e = Expression.Lambda<Func<T,T,T>> (
	        	Expression.Add (paramA, paramB),
	        	Expression.Parameter (typeof(T), "a"),
	        	Expression.Parameter (typeof(T), "b"));

	    	return e.Compile () (a, b);
		}
		\end{minted}
	\end{frame}

\section{Expression trees}
\subsection{Language introspection}
	\begin{frame}
		AST
	\end{frame}
	\begin{frame}
	\end{frame}
\subsection{Cross-compilation}
\subsection{Performance optimization}

\section{Expression trees Mk II}
\subsection{Assignment}
\subsection{Control flow}
\subsection{Member dispatch}

\begin{frame}
	\frametitle{Questions?}
\end{frame}

\end{document}
